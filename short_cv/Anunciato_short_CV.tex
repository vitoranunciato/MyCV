%!TEX TS-program = xelatex
%!TEX encoding = UTF-8 Unicode
% Awesome CV LaTeX Template for CV/Resume
%
% This template has been downloaded from:
% https://github.com/posquit0/Awesome-CV
%
% Author:
% Claud D. Park <posquit0.bj@gmail.com>
% http://www.posquit0.com
%
%
% Adapted to be an Rmarkdown template by Mitchell O'Hara-Wild
% 23 November 2018
%
% Template license:
% CC BY-SA 4.0 (https://creativecommons.org/licenses/by-sa/4.0/)
%
%-------------------------------------------------------------------------------
% CONFIGURATIONS
%-------------------------------------------------------------------------------
% A4 paper size by default, use 'letterpaper' for US letter
\documentclass[11pt, a4paper]{awesome-cv}

% Configure page margins with geometry
\geometry{left=1.4cm, top=.8cm, right=1.4cm, bottom=1.8cm, footskip=.5cm}

% Specify the location of the included fonts
\fontdir[fonts/]

% Color for highlights
% Awesome Colors: awesome-emerald, awesome-skyblue, awesome-red, awesome-pink, awesome-orange
%                 awesome-nephritis, awesome-concrete, awesome-darknight

\colorlet{awesome}{awesome-red}

% Colors for text
% Uncomment if you would like to specify your own color
% \definecolor{darktext}{HTML}{414141}
% \definecolor{text}{HTML}{333333}
% \definecolor{graytext}{HTML}{5D5D5D}
% \definecolor{lighttext}{HTML}{999999}

% Set false if you don't want to highlight section with awesome color
\setbool{acvSectionColorHighlight}{true}

% If you would like to change the social information separator from a pipe (|) to something else
\renewcommand{\acvHeaderSocialSep}{\quad\textbar\quad}

\def\endfirstpage{\newpage}

%-------------------------------------------------------------------------------
%	PERSONAL INFORMATION
%	Comment any of the lines below if they are not required
%-------------------------------------------------------------------------------
% Available options: circle|rectangle,edge/noedge,left/right

\photo{lore.jpg}
\name{Vitor}{Anunciato}

\position{Research scientist}
\address{402 West State Farm Road, 69101 North Platte, US. Nationality:
Brazilian, Birthdate: 10/16/1991}

\mobile{+1 402-853-3893}
\email{\href{mailto:vitor.muller@gmail.com}{\nolinkurl{vitor.muller@gmail.com}}}
\github{vitoranunciato}
\linkedin{vitoranunciato}
\twitter{vitor\_anunciato}

% \gitlab{gitlab-id}
% \stackoverflow{SO-id}{SO-name}
% \skype{skype-id}
% \reddit{reddit-id}

\quote{\textasciitilde{}}

\usepackage{booktabs}

\providecommand{\tightlist}{%
	\setlength{\itemsep}{0pt}\setlength{\parskip}{0pt}}

%------------------------------------------------------------------------------



% Pandoc CSL macros
\newlength{\cslhangindent}
\setlength{\cslhangindent}{1.5em}
\newlength{\csllabelwidth}
\setlength{\csllabelwidth}{3em}
\newenvironment{CSLReferences}[3] % #1 hanging-ident, #2 entry spacing
 {% don't indent paragraphs
  \setlength{\parindent}{0pt}
  % turn on hanging indent if param 1 is 1
  \ifodd #1 \everypar{\setlength{\hangindent}{\cslhangindent}}\ignorespaces\fi
  % set entry spacing
  \ifnum #2 > 0
  \setlength{\parskip}{#2\baselineskip}
  \fi
 }%
 {}
\usepackage{calc}
\newcommand{\CSLBlock}[1]{#1\hfill\break}
\newcommand{\CSLLeftMargin}[1]{\parbox[t]{\csllabelwidth}{#1}}
\newcommand{\CSLRightInline}[1]{\parbox[t]{\linewidth - \csllabelwidth}{#1}}
\newcommand{\CSLIndent}[1]{\hspace{\cslhangindent}#1}

\begin{document}

% Print the header with above personal informations
% Give optional argument to change alignment(C: center, L: left, R: right)
\makecvheader

% Print the footer with 3 arguments(<left>, <center>, <right>)
% Leave any of these blank if they are not needed
% 2019-02-14 Chris Umphlett - add flexibility to the document name in footer, rather than have it be static Curriculum Vitae
\makecvfooter
  {December 07, 2020}
    {Vitor Anunciato~~~·~~~Curriculum Vitae}
  {\thepage}


%-------------------------------------------------------------------------------
%	CV/RESUME CONTENT
%	Each section is imported separately, open each file in turn to modify content
%------------------------------------------------------------------------------



\hypertarget{about-me}{%
\section{About me}\label{about-me}}

I am a Ph.D.~student in Crop Protection (Weed Science), expected to
graduate by March of 2021. My research scope includes tank‑mix
interactions, resistant weed management, and data analysis. I have been
advised by Dr.~Caio A. Carbonari and Co‑advisor by Dr.~Greg R. Kruger.
The title of my thesis for my Master's degree is The Effect of
glyphosate on the growth and reproduction of Digitaria insularis. My
ungraduate focus was in agronomy and I was advised by Dr.~Dionísio L. P.
Gazziero, in the following lines of research: resistance of weeds to
herbicides and weed management in soybeans. I also had experience with
Junior Enterprise and Research as an intern, where I found my passion
for agronomic research. Since then, I feel very keen to pursue a career
in research, science, and industry, dedicating my entire academic life
to contribute to those subjects.

\hypertarget{education}{%
\section{Education}\label{education}}

\begin{cventries}
    \cventry{Ph.D. in Crop Protection}{Universidade Estadual Paulista Júlio de Mesquita Filho}{Botucatu, SP, BR}{Mar. 2018 - Present}{\begin{cvitems}
\item Expected graduation: Mar 2021.
\end{cvitems}}
    \cventry{Master's degree in Crop Protection}{Universidade Estadual Paulista Júlio de Mesquita Filho}{Botucatu, SP, BR}{Mar. 2016 - Mar. 2016}{\vspace{-4mm}}
    \cventry{Environmental management specialization}{École Nationale du Genie de l'Eau et de l'Environnement de Strasbourg}{Strasbourg, AL, FR}{Jul. 2014 - Jul. 2015}{\vspace{-4mm}}
    \cventry{Bachelor degree in Agronomy}{Universidade Estadual de Londrina}{Londrina, PR, BR}{Mar. 2010 - Mar. 2016}{\vspace{-4mm}}
\end{cventries}

\hypertarget{working-experience}{%
\section{Working Experience}\label{working-experience}}

\begin{cventries}
    \cventry{Researcher scholar (Pesticide Application Technology Laboratory)}{University of Nebraska-Lincoln}{North Platte, NE, US}{Mar. 2020 - to present}{\begin{cvitems}
\item Functions: Field and greenhouse trials, statistical analysis, and articles productions
\item Results: Helped in Dicamba field drifty studies, studies of weed management, and physical-chemical tank-mix interactions with technology application parameters.
\item Also the production of my Dissertation still in progress: Statistical analysis of herbicide interactions in the tank mixture.
\end{cvitems}}
    \cventry{Researcher scholar (Crop Protection)}{Universidade Estadual Paulista Júlio de Mesquita Filho}{Botucatu, SP, BR}{Mar. 2016 - Feb. 2020}{\begin{cvitems}
\item Functions: Field and greenhouse trials, statistical analysis, and articles productions
\item Results: Conducted studies of weed management in Brazil and the physical properties of adjuvants in an herbicide tank mixture, resulting in five articles and one book chapter, my thesis, and also various abstracts and presentations at scientific meetings.
\end{cvitems}}
    \cventry{Consulting assistant}{SIGA Consultoria agrícola}{Londrina, PR, BR}{Jun. 2015 - Feb. 2016}{\begin{cvitems}
\item Functions: Monitoring of soybean diseases using the SIGA spore collector, monitoring of agricultural pests, and positioning of pesticides.
\item Results: In addition to the field and laboratory activities, I also created the company's visual identity, developing document templates and PowerPoint presentations.
\end{cvitems}}
    \cventry{Research scholar (Système d'Analyse Régionale des Risques Agroclimatologiques)}{CIRAD Département Performances des systèmes de production et de transformation tropicaux}{Montpellier, LR, FR}{Feb. 2014 - Jun. 2016}{\begin{cvitems}
\item Functions: Improvements in tropical agriculture model SARRA-H
\item Results: I adapted the model for the Brazilian scenario, I also translated the site and user manuals for users of the Sarra-H model from French into Portuguese, facilitating its use by Portuguese speakers.
\end{cvitems}}
    \cventry{Internship (Crop protection and Soil Management )}{Coordenação de Aperfeiçoamento de Pessoal de Nível Superior}{Londrina, PR, BR}{Mar. 2010 - Jun. 2013}{\begin{cvitems}
\item Functions: Develop research projects, implement them, and carry out completion reports. As well as assisting graduate students in their research projects.
\item Results: I assisted in weed control and cover crop management projects, resulting in two book chapters, an article, and some abstracts and presentations at scientific meetings.
\end{cvitems}}
    \cventry{Assistant | Marketing Director | Projects Director}{CONSOAGRO - Empresa Jr. de Agronomia da Universidade Estadual de Londrina}{Londrina, PR, BR}{Mar. 2010 - Apr. 2014}{\begin{cvitems}
\item Functions: Assistance in agronomic events and consulting projects. Represent the company before society, marketing coordination, and project management.
\item Results: My team developed all the marketing, social media, website, and standard arts of the company. In the project section, my team structured the company's social projects and we were also responsible for the first technical consultancy carried out by the junior company.
\end{cvitems}}
\end{cventries}

\hypertarget{publications}{%
\section{Publications}\label{publications}}

\hypertarget{journal-papers}{%
\subsection{Journal Papers}\label{journal-papers}}

\begingroup
\setlength{\parindent}{-0.5in}
\setlength{\leftskip}{0.5in}

\hypertarget{refs_journals}{}
\leavevmode\hypertarget{ref-bianchi2020effects}{}%
Bianchi, L., Anunciato, V. M., Gazola, T., Perissato, S. M., Carvalho
Dias, R. de, Tropaldi, L., Carbonari, C. A., \& Velini, E. D. (2020).
Effects of glyphosate and clethodim alone and in mixture in sourgrass
(digitaria insularis). \emph{Crop Protection}, \emph{138}, 105322.

\leavevmode\hypertarget{ref-doi:10.1080ux2f03601234.2020.1853459}{}%
Bianchi, L., Perissato, S. M., Anunciato, V. M., Dias, R. C., Gomes, D.
M., Carbonari, C. A., \& Velini, E. D. (2020). Stimulation action of
mefenpyr-diethyl on soybean, wheat, and signal grass plants.
\emph{Journal of Environmental Science and Health, Part B}, \emph{0}(0),
1--5. \url{https://doi.org/10.1080/03601234.2020.1853459}

\leavevmode\hypertarget{ref-de2020efeito}{}%
Carvalho Dias, R. de, Tropaldi, L., Bianchi, L., Anunciato, V. M.,
Gomes, D. M., Carbonari, C. A., \& Velini, E. D. (2020). EFEITO DO
SEL{Ê}NIO COMO PROTETOR QU{Í}MICO NA SELETIVIDADE INICIAL DE HERBICIDAS
APLICADOS EM p{Ó}s EMERG{Ê}NCIA DE urochloa decumbens. \emph{Revista
Brasileira de Herbicidas}, \emph{19}(2), 710--711.

\leavevmode\hypertarget{ref-dias2020herbicides}{}%
Dias, R. C., Gomes, D. M., Anunciato, V. M., Bianchi, L., Carbonari, C.
A., \& Velini, E. D. (2020). Herbicides selectivity on seedlings of
white leadtree (leucaena leucocephala). \emph{Cient{ı́}fica},
\emph{48}(1), 56--66.

\leavevmode\hypertarget{ref-dias2019seleccao}{}%
Dias, R. C., Gomes, D. M., Anunciato, V. M., Bianchi, L., Simões, P. S.,
Carbonari, C. A., \& Velini, E. D. (2019). Sele{ç}{ã}o de esp{é}cies
bioindicadoras para o herbicida indaziflam. \emph{Revista Brasileira de
Herbicidas}, \emph{18}(2), 650--651.

\leavevmode\hypertarget{ref-meschede2015absorption}{}%
Meschede, D. K., Silva, F. E. da, Fernandes, D., Olveira, E. C., Moraes
Gomes, M. de, \& Anunciato, V. M. (2015). Absorption and translocation
tolerance of glyphosate. \emph{African Journal of Agricultural
Research}, \emph{10}(52), 4738--4747.

\endgroup

\hypertarget{book-chapters}{%
\subsection{Book chapters}\label{book-chapters}}

Gazola, T. ; Anunciato, V. M. ; Dias, R. C. ; Bianchi, L. ; Moraes, C.
P. ; Ferrari, J. L. . MECANISMOS DE AÇÃO DE HERBICIDAS E IDENTIFICAÇÃO
DE PLANTAS DANINHAS. In: Edson Luiz Lopes Baldin; Adriana Zanin Kronka;
Ivana Fernandes da Silva. (Org.). INOVAÇÕES EM MANEJO FITOSSANITÁRIO.
1ed.Botucatu: Fundação de Estudos e Pesquisas Agrícolas e Florestais,
2017, v. 1, p.~5-225.

Gazziero, D.L.P.; Santos, A.M.B.; Adegas, F.S.; Anunciato, V.M.. LOSNA
BRANCA RESISTENTE AO HERBICIDAS INIBIDORES DA ALS. In: Vargas, L. ,
Agostinetto, D.. (Org.). Resistência de Plantas Daninhas a Herbicidas no
Brasil. 1ed.Pelotas: Pelotas: UFPel, 2014, v. 1, p.~7-383.

Gazziero, D.L.P.; Santos, A.M.B.; Adegas, F.S.; Anunciato, V.M.; Karam,
D.. BIÓTIPOS DE PICÃO-PRETO(Bidens subalternans) RESISTENTES A ATRAZINE.
In: Vargas, L. , Agostinetto, D.. (Org.). Resistencia de plantas
daninhas a herbicidas no brasil. 1ed.Pelotas: Pelotas: UFPel, 2014, v.
1, p.~7-383.

\pagebreak

\hypertarget{skills}{%
\section{Skills}\label{skills}}

\begin{cvskills}
  \cvskill
    {Data Science}
    {R (advanced)}

  \cvskill
    {Reproducible Report}
    {Markdown/Rmarkdown, R shiny apps}

  \cvskill
    {Quantitative Research}
    {Dose response curve, Colby analysis, t-test, ANOVAs/ANCOVAs/MANOVAs/MANCOVAs,\newline
    Regressions, Factor Analysis, PCA, Unsupervised/Supervised Machine Learning}

  \cvskill
    {Languages}
    {Portuguese/English/French}
\end{cvskills}

\hypertarget{service}{%
\section{Service}\label{service}}

\begin{cventries}
    \cventry{Reviewer}{Crop Protection}{Elsevier Ltd.}{2020}{\vspace{-4mm}}
    \cventry{Reviewer}{Revista Brasileira de Herbicidas}{ISSN 2236-1065}{2018}{\vspace{-4mm}}
    \cventry{Host}{PapoAgro podcast}{Online}{2019}{\vspace{-4mm}}
    \cventry{Outreach}{Serial Weed Killer}{serialweedkiller.netlify.app/}{2019}{\vspace{-4mm}}
\end{cventries}

\hypertarget{references}{%
\section{References}\label{references}}

\begin{itemize}
\item
  Associate Professor \href{gkruger2@unl.edu}{Greg R. Kruger},
  University of Nebraska-Lincoln. West Central Research and Extension
  Center. 402 West State Farm Road, North Platte, NE, 69101. Work: +1
  308-696-6715,
  \href{mailto:greg.kruger@unl.edu}{\nolinkurl{greg.kruger@unl.edu}}
\item
  Professor \href{carbonari@fca.unesp.br}{Caio A. Carbonari},
  Universidade Estadual Paulista Júlio de Mesquita Filho, Fazenda
  Experimental Lageado, CEP 18610-034, Botucatu, SP, Brazil. Work: +55
  14 991145170,
  \href{mailto:carbonari@fca.unesp.br}{\nolinkurl{carbonari@fca.unesp.br}}
\item
  Researcher \href{dionisio.gazziero@embrapa.br}{Dionísio L. P.
  Gazziero}, Empresa Brasileira de Pesquisa Agropecuária (EMBRAPA Soja),
  Rd Carlos João Strass, s/nº CEP: 86001-970, Londrina, PR, Brazil.
  Work: +55 43 99948062,
  \href{mailto:dionisio.gazziero@embrapa.br}{\nolinkurl{dionisio.gazziero@embrapa.br}}.
\end{itemize}

\includegraphics{../data/signature.png}\\

\end{document}
